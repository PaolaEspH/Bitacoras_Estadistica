\documentclass{book}

\usepackage[spanish]{babel}


\usepackage[letterpaper,top=2cm,bottom=2cm,left=3cm,right=3cm,marginparwidth=1.75cm]{geometry}

\usepackage{amsmath}
\usepackage{graphicx}
\usepackage[colorlinks=true, allcolors=blue]{hyperref}

\title{Accidentes de Tráfico en Nueva Zelanda}
\author{Paola Espinoza Hernández, Jeikel Navarro y Gabriel Sanabria Alvarado}

\begin{document}
\maketitle


\chapter*{Bitácora 1}

\section{Planificación}

\subsection{Pregunta de investigación}
¿Cuáles variables tienen mayor impacto en la severidad de los accidentes de tráfico en Nueva Zelanda?

\subsubsection{Definición de la idea}

\subsubsection{Conceptualización de la idea}

\subsubsection{Identificación de tensiones}

\subsubsection{Reformulación de la idea en modo de pregunta}

\subsubsection{Argumentación de la pregunta}

\subsubsection{Argumentación a través de datos}

\section{Revisión bibliográfica}

\subsubsection{Ficha 1: Estimación de probabilidades de accidentes basada en estados de tráfico en autopistas urbanas}
\textbf{Autor:} Cristian Nicolás Zúñiga González

\textbf{Año:} 2014

\textbf{Tema:} Impacto de los estados de tráfico en la ocurrencia y severidad de accidentes

\textbf{Forma de organizarlo:}

\begin{itemize}
\setlength{\itemindent}{0.5in}
    \item \textbf{Cronológico:} 2012 y primera semana del 2013
    \item \textbf{Metodológico:} Calibración de modelos de elección discreta
    \item \textbf{Temático:} Modelos de elección discreta
    \item \textbf{Teoría:} Modelos para estimar el impacto de diversas variables en la ocurrencia y severidad de accidentes
\end{itemize}

\textbf{Resumen en una oración:} La siniestralidad y severidad de un accidente se ve afectada por el estado del tráfico.

\textbf{Argumento central:} Los estados de transición generan más riesgo de ocurrencia para todos los accidentes.

\textbf{Problemas con el argumento o el tema:} Aunque todos los indicadores mejoran cuando la muestra crece, presentan problemas ante una especificación más desagregada. La vía de estudio, Autopista Central de Santiago en Chile, no posee características desafiantes relacionadas a las condiciones climáticas o la geometría.

\textbf{Resumen en un párrafo:} Entre 2006 y 2012, la siniestralidad en la Autopista Central de Santiago tuvo una  tasa de aumento mayor a la tasa de crecimiento de vehículos, con accidentes recurrentes en lugares o condiciones similares. Se ha detectado anteriormente una relación entre el estado de tráfico y condiciones climáticas y la ocurrencia de accidentes, pero sin discriminar la severidad. Este estudio no solo clasifica los accidentes según su gravedad, sino que también amplía la evaluación del costo social al incluir los efectos de la congestión por bloqueos viales. Se concluye que factores como pendientes descendentes, cambios bruscos de velocidad, alta desviación estándar de velocidad y tramos largos incrementan la probabilidad de accidentes. Además, los estados de tráfico más riesgosos son "cuello de botella", "congestión" y, especialmente, "final de cola", este último asociado a una mayor probabilidad de heridos en pendientes descendentes. Por último, la variabilidad en la velocidad y ajustes frecuentes en ella también elevan el riesgo, evidenciando que la combinación de dinámicas de flujo y diseño vial influye en la siniestralidad.

\subsubsection{Ficha 2: Título}
\textbf{Autor:}

\textbf{Año:}

\textbf{Tema:}

\textbf{Forma de organizarlo:}

\begin{itemize}
\setlength{\itemindent}{0.5in}
    \item \textbf{Cronológico:} 
    \item \textbf{Metodológico:} 
    \item \textbf{Temático:} 
    \item \textbf{Teoría:} 
\end{itemize}

\textbf{Resumen en una oración:}

\textbf{Argumento central:}

\textbf{Problemas con el argumento o el tema:}

\textbf{Resumen en un párrafo:}

\subsubsection{Ficha 3: Título}
\textbf{Autor:}

\textbf{Año:}

\textbf{Tema:}

\textbf{Forma de organizarlo:}

\begin{itemize}
\setlength{\itemindent}{0.5in}
    \item \textbf{Cronológico:} 
    \item \textbf{Metodológico:} 
    \item \textbf{Temático:} 
    \item \textbf{Teoría:} 
\end{itemize}

\textbf{Resumen en una oración:}

\textbf{Argumento central:}

\textbf{Problemas con el argumento o el tema:}

\textbf{Resumen en un párrafo:}

\section{UVE de Gowin}

\begin{figure}
\centering
\includegraphics[width=0.25\linewidth]{frog.jpg}
\caption{\label{fig:frog}This frog was uploaded via the file-tree menu.}
\end{figure}

\bibliographystyle{alpha}
\bibliography{sample}

\end{document}