\documentclass{book}

\usepackage[spanish]{babel}


\usepackage[letterpaper,top=2cm,bottom=2cm,left=3cm,right=3cm,marginparwidth=1.75cm]{geometry}

\usepackage{amsmath}
\usepackage{graphicx}
\usepackage[colorlinks=true, allcolors=blue]{hyperref}

\title{Accidentes de Tráfico en Nueva Zelanda}
\author{Paola Espinoza Hernández, Jeikel Navarro y Gabriel Sanabria Alvarado}

\begin{document}
\maketitle


\chapter*{Bitácora 1}

\section{Planificación}

\subsection{Pregunta de investigación}
¿Cuáles variables tienen mayor impacto en la severidad de los accidentes de tráfico en Nueva Zelanda?

\subsubsection{Definición de la idea}

\subsubsection{Conceptualización de la idea}

\subsubsection{Identificación de tensiones}

\subsubsection{Reformulación de la idea en modo de pregunta}

\subsubsection{Argumentación de la pregunta}

\subsubsection{Argumentación a través de datos}

\section{Revisión bibliográfica}

\subsubsection{Estimación de probabilidades de accidentes basada en estados de tráfico en autopistas urbanas}
\textbf{Autor:} Cristian Nicolás Zúñiga González

\textbf{Año:} 2014

\textbf{Tema:} Impacto de los estados de tráfico en la ocurrencia y severidad de accidentes

\textbf{Forma de organizarlo:}

\begin{itemize}
\setlength{\itemindent}{0.5in}
    \item \textbf{Cronológico:} datos del 2012 y primera semana del 2013
    \item \textbf{Metodológico:} Calibración de modelos de elección discreta
    \item \textbf{Temático:} Modelos de elección discreta
    \item \textbf{Teoría:} Modelos para estimar el impacto de diversas variables en la ocurrencia y severidad de accidentes
\end{itemize}

\textbf{Resumen en una oración:} La siniestralidad y severidad de un accidente se ve afectada por el estado del tráfico.

\textbf{Argumento central:} Los estados de transición generan más riesgo de ocurrencia para todos los accidentes.

\textbf{Problemas con el argumento o el tema:} Aunque todos los indicadores mejoran cuando la muestra crece, presentan problemas ante una especificación más desagregada. La vía de estudio, Autopista Central de Santiago en Chile, no posee características desafiantes relacionadas a las condiciones climáticas o la geometría.

\textbf{Resumen en un párrafo:} Entre 2006 y 2012, la siniestralidad en la Autopista Central de Santiago tuvo una  tasa de aumento mayor a la tasa de crecimiento de vehículos, con accidentes recurrentes en lugares o condiciones similares. Se ha detectado anteriormente una relación entre el estado de tráfico y condiciones climáticas y la ocurrencia de accidentes, pero sin discriminar la severidad. Este estudio no solo clasifica los accidentes según su gravedad, sino que también amplía la evaluación del costo social al incluir los efectos de la congestión por bloqueos viales. Se concluye que factores como pendientes descendentes, cambios bruscos de velocidad, alta desviación estándar de velocidad y tramos largos incrementan la probabilidad de accidentes. Además, los estados de tráfico más riesgosos son "cuello de botella", "congestión" y, especialmente, "final de cola", este último asociado a una mayor probabilidad de heridos en pendientes descendentes. Por último, la variabilidad en la velocidad y ajustes frecuentes en ella también elevan el riesgo, evidenciando que la combinación de dinámicas de flujo y diseño vial influye en la siniestralidad.

\subsubsection{Analysis of the factors affecting the severity of two-vehicle crashes}
\textbf{Autor:} Alejandro Ángel, Mark Hickman

\textbf{Año:} 2008

\textbf{Tema:} Factores que afectan la severidad de accidentes vehiculares

\textbf{Forma de organizarlo:}

\begin{itemize}
\setlength{\itemindent}{0.5in}
    \item \textbf{Cronológico:} 2008, con datos de 1995 a 2004
    \item \textbf{Metodológico:} Modelo logit multinomial (para severidad de heridas), regresión lineal (costos)
    \item \textbf{Temático:} Modelación
    \item \textbf{Teoría:} Modelación de impacto en severidad de accidentes
\end{itemize}

\textbf{Resumen en una oración:} El tipo de choque tiene un impacto muy significativo en la severidad del accidente.

\textbf{Argumento central:} Existen muchos factores, tanto de comportamiento personal, características vehículares y de tipo de choque, que afectan la severidad de los accidentes de tráfico.

\textbf{Problemas con el argumento o el tema:} Se considera únicamente el costo de las heridas, no el costo social de la obstrucción vial causada por el choque. El modelo lineal de costos podría ser una hipersimplificación, y no reflejar el verdadero costo.

\textbf{Resumen en un párrafo:} Los accidentes vehículares poseen muchos factores que pueden afectar el resultado de este. Entre estos, destacan el tipo de choque, como el choque de cabeza, que puede incrementar más de 56 veces la fatalidad comparada al choque por la parte trasera del vehículo. Además, el uso de cinturón de seguridad disminuye la severidad del choque, mientras que el estar alcoholizado la aumenta. Adicionalmente, es importante diferenciar las consecuencias para cada participante, pues en los choques de vehículos grandes, estos reducen la severidad para quien los ocupa, pero aumenta la de los demás participantes. 

\subsubsection{Variables predictoras de víctimas graves, críticas o fallecidas en los accidentes de tráfico en Extremadura}
\textbf{Autores:} José Antonio Morales Gabardino, Laura Redondo-Lobato, Francisco Buitrago-Ramírez

\textbf{Año:} 2019

\textbf{Tema:} Variables predictoras de la severidad de accidentes de tráfico

\textbf{Forma de organizarlo:}

\begin{itemize}
\setlength{\itemindent}{0.5in}
    \item \textbf{Cronológico:} datos del 2012 al 1015
    \item \textbf{Metodológico:} Regresión logística binaria
    \item \textbf{Temático:} Predicción de severidad de accidentes vehiculares
    \item \textbf{Teoría:} Factores de riesgo en accidentes viales
\end{itemize}

\textbf{Resumen en una oración:} Los accidentes interurbanos, la edad y la zona aumentan la gravedad y mortalidad vial.

\textbf{Argumento central:} Los accidentes de tránsito interurbanos tienen mayor gravedad y mortalidad que los urbanos, siendo el tipo de accidente, la edad y la ubicación importantes factores predictivos de su severidad.

\textbf{Problemas con el argumento o el tema:} La base de datos utilizada no posee registro del sexo de los accidentados, variabe que parece estar ausente en varios otros estudios. Estos datos tampoco incluyen la evolución final de los pacientes, de modo que no se puede evaluar la repercusión socioeconómica de estos accidentes, y la posible subestimación de fallecidos, pues solo se registran los fallecidos al momento de la atención inicial.

\textbf{Resumen en un párrafo:} En el artículo se analiza si el tipo de accidente (urbano o interurbano), la edad o la atención médica influyen en el resultado del mismo. Se concluye que la mayoría de los accidentes son de tipo interurbano y que estos presentan un mayor porcentaje de gravedad en comparación con los ocurridos en zonas urbanas.El tipo de accidente destaca como una variable predictora relevante, ya que puede incrementar el riesgo de fallecimiento en un $74,5\%$.Asimismo, tanto la edad como las zonas de influencia aumentan la probabilidad de accidentes graves o críticos, por lo que estos factores también deberían considerarse para predecir la severidad de los siniestros viales.

\section{UVE de Gowin}

\begin{figure}
\centering
\includegraphics[width=0.25\linewidth]{frog.jpg}
\caption{\label{fig:frog}This frog was uploaded via the file-tree menu.}
\end{figure}

\bibliographystyle{plain}
\bibliography{referencias}
\nocite{*}

\end{document}